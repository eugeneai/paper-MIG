
% Базовый проект
% Реализация инструментальных средств преобразования онтологических описаний в структуры UML с последующей их интерпретаций в виде программного кода компонент ИС и АРМ.

% РФФИ

% Разработка специализированной системы трансформации платформонезависимой модели, ориентированной на поддержку современных средств разработки интернет-приложений, функционирующих в среде Linked Open Data.



\documentclass[12pt]{article}
\RequirePackage[a4]{}
\RequirePackage{luatextra}
\RequirePackage[a4paper,margin=2cm,includeheadfoot,nofoot,showframe]{geometry}
\RequirePackage{polyglossia}
\setmainlanguage{russian}
\setotherlanguage{english}
\setkeys{russian}{babelshorthands=true}


\setmainfont{Times New Roman}
\setromanfont{Times New Roman}
\setsansfont{Arial}
\setmonofont{Courier New}


\newfontfamily{\cyrillicfont}{Times New Roman}
\newfontfamily{\cyrillicfontrm}{Times New Roman}
\newfontfamily{\cyrillicfonttt}{Courier New}
\newfontfamily{\cyrillicfontsf}{Arial}


\addto\captionsrussian{%
  \renewcommand{\figurename}{Рис.}%
  \renewcommand{\tablename}{Табл.}%
}
\setlength{\parindent}{1cm}

% Поля страницы: верхнее, нижнее, левое, правое – 2 см.

\setmainfont{Times New Roman}

\begin{document}
УДК 004.4'244
\begin{flushright}\itshape{}
  \textbf{Е.А.~Черкашин}\\
  кандидат технических наук, доцент\\
  г.~Иркутск, Институт динамики систем и теории управления СО РАН
\end{flushright}
\begin{center}
  \Large\bfseries Реализация трансформации моделей на основе объектно-ориентированного логического программирования
\end{center}

\textbf{Аннотация.} Текс аннотации

\textbf{Ключвые слова.} ...

ТЕкст...

\renewcommand\bibname{Литература}
\begin{thebibliography}{99}
\bibitem{b1} b1
\end{thebibliography}

\end{document}










%%% Local Variables:
%%% mode: latex
%%% TeX-master: "cherkashin-trans-with-LOOP.tex"
%%% End:
