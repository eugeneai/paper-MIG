
% Базовый проект
% Реализация инструментальных средств преобразования онтологических описаний в структуры UML с последующей их интерпретаций в виде программного кода компонент ИС и АРМ.

% РФФИ

% Разработка специализированной системы трансформации платформонезависимой модели, ориентированной на поддержку современных средств разработки интернет-приложений, функционирующих в среде Linked Open Data.



\documentclass[12pt]{article}
\RequirePackage[a4]{}
\RequirePackage{luatextra}
\RequirePackage[a4paper,margin=2cm,includeheadfoot,nofoot,showframe]{geometry}
\RequirePackage{polyglossia}
\setmainlanguage{russian}
\setotherlanguage{english}
\setkeys{russian}{babelshorthands=true}


\setmainfont{Times New Roman}
\setromanfont{Times New Roman}
\setsansfont{Arial}
\setmonofont{Courier New}


\newfontfamily{\cyrillicfont}{Times New Roman}
\newfontfamily{\cyrillicfontrm}{Times New Roman}
\newfontfamily{\cyrillicfonttt}{Courier New}
\newfontfamily{\cyrillicfontsf}{Arial}


\addto\captionsrussian{%
  \renewcommand{\figurename}{Рис.}%
  \renewcommand{\tablename}{Табл.}%
}
\setlength{\parindent}{1cm}

% Поля страницы: верхнее, нижнее, левое, правое – 2 см.

\setmainfont{Times New Roman}

\begin{document}
УДК 004.4'244
\begin{flushright}\itshape{}
  \textbf{Е.А.~Черкашин}\\
  кандидат технических наук, доцент\\
  г.~Иркутск, Институт динамики систем и теории управления СО РАН
\end{flushright}
\begin{center}
  \Large\bfseries Реализация трансформации моделей на основе объектно-ориентированного логического программирования
\end{center}

\textbf{Аннотация.}

\textbf{Ключвые слова.} ...

Существуют предметные области, обладающие высокой динамикой своих свойств и структуры..... пример. .. Даже в случае стабильной структуру предметной области существуют схемы организации разработки, где необходима постоянный возврат на ранние этапы жизненного цикла программного обеспечения, например, при применении экстремального и agile-программирования (``проворного'' программирования).  В таких условиях очень важно быстро вносить изменения, внесенные в дизайн системы, на уровень исходного кода подсистем, причем необходимо, чтобы изменения в разные подсистемы не противоречили друг другу.  Анализ процесса реализации дизайнерских решений для стандартных моделей подсистем, например, UML, для конкретных программных платформ показывает, что ..... повторяется....стандартизован.  Повторяющиеся операции имеет смысл автоматизировать, однако автоматизация творческой деятельности является нетривиальной задачей.

Методы автоматизации вышеуказанного вида основываются на анализе структур объектов и отношений между ними в исходных моделях и последующим порождением производных моделей и исходного кода.  Такие трансформации являются предметом изучения \emph{инженерии программного обеспечения (ПО), основанного на моделировании} (Model Driven Engineering, MDE), относительно нового направления разработки ПО.  Одной из основных и сложных задач, решаемой в MDE, является распространение изменений (change propagation).  В интуитивной постановке задача состоит в том, чтобы сравнить две версии модели и внести соответствующие изменения в остальные модели.  В идеале, изменения могут быть внесены в любую модель (модель любого уровня абстракции, включая сгенерированный ранее исходный код) и должны быть распространены (propagated) в остальные модели.

Более простой задачей в MDE - это порождение исходного кода подсистем по набору исходных абстрактных моделей.  В качестве моделей могут выступать структуры базы данных, загруженная и проанализированная ....., или UML и т.п...  Первый подход исследуется в отдельном направлении - разработке, основанной на данных (Data Driven Engineering, DDE), второй - архитектуры, основанной на моделировании (Model Driven Architecure, MDA).  MDA является предметом нашего НИРОКР.  MDA реалузет тресформации поэтапно.  Исходный код подсистем порождается на основе его \emph{платформонезависимой модели} (Platform Independent Model, PIM), представляющая исходный код таким образом, чтобы его можно было алогитмически из нее сгенерировать.  PIM порождается из модели программной системы, где не представлены ньансы платформы реализации, платформонезависимой модели (Platform Independent Model, PIM), ее задача представлять структуры данных и высислительный процесс в абстрактном виде.  Эта модель, в свою очередь, польностью или частично порождается из вычислительно-независимой модели (Computationally Independent Model, CIM), которая, по сути, абстрактная модель предметной области.  Трансформации реализуются на основе модели платформы (Platform Model, PM), представляющей свойства платформы и методики реализации, а также традиции программирования конкретной группы разработчиков.  Именно наличие модели платформы отличает MDA и использованные в 90-х годах CASE-подхода (Computer-Aided Software Engineering).  В основе идеи MDA находится  возможность совершенствования модели транформации параллельно с основной разработкой программной системой.

Основная задача данного исследования - разработка инструментальных средств представления CIM, PIM, PSM и PD, позволяющие создавать каркасы программных систем и, в частности, информационных систем, разработка последних весьма популярна и также наиболее стандартизирована плоть до отдельных сценариев.  В данной статье представлена разработанная нами методика использования объектно-ориентированного логичексого программирования и средств Семантического веба для определение и реализации сценариев трансформации.

% ORM, РБД, ООП


\section{Трансформация диаграммы классов в ORM}
\label{sec:trorm}

Выполнение трансформации проще всего рассмотреть на примере преобразования диаграммы классов UML в текстовое представления модели.

Дизайнер структур данных може тользоваться различными способами предстваления модели, в частности, изображать все классы, интерфейсы, экземпляры и т.п. на одной диаграмме или реализовывать какую-либо модульность: классы бизнес-логики помещаются на отдельную отдельную, а классы ORM на сдругую.

\section{Использование средств моделирования для расширения функций ORM}

Популярные ORM требуют, чтобы структура ORM-модели соответствовала свойствам реляционной модели.  Например, синтетический ключевой атрибут, обычно называемый \texttt{id}, обязан присутствовать во всех классах модели.  В ООП подобная проблема избыточности решается при помощи наследования атрибутов у родительского класса. Однако, не все ORM позволяют создавать абстрактные классы, потомки которых - модели ORM.  Примером такой ORM является популярная в среде Python библиотека SQLAlchemy.  При помощи трансформации моделей эта задача легко решается через реализацию наследования от класса, помеченного стереотипом \texttt{<<abstract>>} ORM-классом.  Более того, можно также решать задачи конфигурирования процесса трансформации при помощи специальных стереотипов, реализовывать отношения между классами-моделями разными способами, дополнять ORM новыми возможностями.

\renewcommand\bibname{Литература}
\begin{thebibliography}{99}
\bibitem{b1} b1
\end{thebibliography}

\end{document}










%%% Local Variables:
%%% mode: latex
%%% TeX-master: "cherkashin-trans-with-LOOP.tex"
%%% End:
